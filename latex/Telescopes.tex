
\AgdaHide{
\begin{code}\>\<%
\\
\>\AgdaSymbol{\{-\#} \AgdaKeyword{OPTIONS} --type-in-type --no-positivity-check --no-termination-check \AgdaSymbol{\#-\}}\<%
\\
%
\\
\>\AgdaKeyword{module} \AgdaModule{Telescopes} \AgdaKeyword{where}\<%
\\
%
\\
\>\AgdaKeyword{open} \AgdaKeyword{import} \AgdaModule{Relation.Binary.PropositionalEquality} \<[50]%
\>[50]\<%
\\
\>\AgdaKeyword{open} \AgdaKeyword{import} \AgdaModule{Data.Product} \AgdaKeyword{renaming} \AgdaSymbol{(}\_,\_ \AgdaSymbol{to} \_,,\_\AgdaSymbol{)}\<%
\\
\>\AgdaKeyword{open} \AgdaKeyword{import} \AgdaModule{Data.Nat}\<%
\\
%
\\
\>\AgdaKeyword{open} \AgdaKeyword{import} \AgdaModule{BasicSyntax} \AgdaKeyword{renaming} \AgdaSymbol{(}\_∾\_ \AgdaSymbol{to} htrans\AgdaSymbol{)}\<%
\\
\>\AgdaKeyword{open} \AgdaKeyword{import} \AgdaModule{BasicSyntax2}\<%
\\
\>\AgdaKeyword{open} \AgdaKeyword{import} \AgdaModule{Suspension}\<%
\\
\>\AgdaKeyword{open} \AgdaKeyword{import} \AgdaModule{GroupoidLaws}\<%
\\
\>\<\end{code}
}

\subsection{Higher Structure}
%
In this section we propose how also higher groupoid structure can be
introduced in the syntactical framework. We use the more abstract
language of category theory to communicate the gist of the
construction leaving the tedious formalisation for future work. To this
end note that contexts and context morphisms form a category up to
definitional quality. Because equality of contexts is decidable we
may assume \textsf{UIP} on context morphisms and we are therefore
working in a honest 1-category where equality of arrows is
definitional equality of context morphisms. This category will be
denoted $\mathsf{Con}$.

\subsubsection{Identities}
\label{sec:spans}

For each type of level $n\in \mathbf{N}$, we have defined in Section \ref{sec:replacement} a context called
\emph{span} which has $2n+1$ variables
which except for the top level, $n$, there are two variables on each
level whose type is the equality type of the two variables on the
level below, except for the bottom-level variables which are of type $\ast$. 
We call denote a span of any type of level $n$, $S_n$. Note that all
such spans are isomorphic. 

%p
In each case we call the last variable the
\emph{peak}. Note that each $S_n$ is contractible because it is a
suspension of a contractible context. We call the proof of
contractibility of $S_n$ $\mathsf{is\text{-}contr}~S_n$. 

In each $S_n$ define the type $\sigma_n$ as $x_{2n-2}\,=_\mathsf{h}\,x_{2n-1}$. It is the type of the top variable. We are going to show that the following 
\[
S_0 \three/<-`>`<-/^{s_0}|{i_0}_{t_0} S_1 \three/<-`>`<-/^{s_1}|{i_1}_{t_1} S_2 \cdots S_n
\three/<-`>`<-/^{s_n}|{i_n}_{t_n} S_{n+1} \cdots
\]
is a \emph{reflexive globular object} in $\mathsf{Con}$. I.e. we define
morphisms $s_n$, $t_n$, $i_n$ between spans that it satisfy the
following usual \emph{globular identities}:
\begin{equation}\label{eq:glob}
\begin{array}{rl}
s_n t_{n+1}&~=~s_n s_{n+1}\\
t_n t_{n+1}&~=~t_n s_{n+1}
\end{array}
\end{equation}
% 
together with
\begin{equation}
  \label{eq:refl-glob}
  s_n i_n ~= ~\mathsf{id}~=~ t_ni_n
\end{equation}

To this end, for each $n$, define context morphisms $s_n, t_n  : S_{n+1}
\Rightarrow S_n$ by the substitutions
\begin{align*}
s_n &~=~& x_k &~\mapsfrom~x_k &k < 2n\\
&&x_{2n} &~\mapsfrom~x_{2n} \\
t_n &~=~& x_k &~\mapsfrom~x_k &k < 2n\\
&&x_{2n} &~\mapsfrom~2n+1
\end{align*}
%
In words, $s_n$ forgets the last two variables of $S_{n+1}$ and
$t_n$ forgets the peak and its domain. It's easy to check that $s$ and
$t$ indeed satisfy \eqref{eq:glob}.


In order to define $i_n :  S_n \Rightarrow S_{n+1}$, 
we must consider stalks (see Section \ref{sec:susp-and-repl}), which
are contexts, hereby denoted $\overline{S_n}$, which are like
$S_n$ without the last variable, together with types
$\overline{\sigma}_n$ which are like $\sigma_n$ but considered in the
smaller context. 

%
For each $n$ there is a context morphism $\overline{i_n}: S_n
\Rightarrow \overline{S}_{n+1}$ defined by
\begin{align*}
\overline{i_n} &~=~& x_k &~\mapsfrom ~ x_k & k \le 2n\\
&& x_{2n+1} &~\mapsfrom~x_{2n}
\end{align*}
%
The substitution of $\overline{\sigma}_{n+1}$ along $\overline{i_n}$,
$\overline{i_n}[\overline{\sigma}_{n+1}]_\mathsf{T}$, is equal to $x_{2n}
=_\mathsf{h} x_{2n}$ in $S_n$. So in order to extend $\overline{i_n}$ to
$i_n : S_n \Rightarrow S_{n+1}$ we must define a term in
$\overline{i_n}[\overline{\sigma}_{n+1}]_\mathsf{T}$.  We can
readily do that by $\mathsf{coh}$:
\[
i_n~ = ~ \overline{i_n}\, ,
\,\mathsf{coh}~(\mathsf{IdCm\,S_n})~(x_{2n}\,=_\mathsf{h}\,x_{2n})~(\mathsf{is\text{-}contr}~S_n)
\]
\noindent It's easy to check that $i_n$ satisfies \eqref{eq:refl-glob}. 

For each $n$ consider the chain 
\[i_0^n \quad \equiv \quad S_0 \to^{i_0} S_1 \to^{i_1} S_2 \to \cdots \to^{i_{n-1}} S_n\]
%
The substitution
$\sigma_n[i_0]_\mathsf{T}\cdots[i_{n-2}]_\mathsf{T}[i_{n-1}]_\mathsf{T}
\equiv \sigma_n[i_0^n]_\mathsf{T}$ is
a type, $\lambda_n$, in $S_0$.
We call $\lambda_n$ the \emph{n\text{-}iterated loop type} on 
$x_0$. The term $S_0 \vdash (\mathsf{var}~x_{2n}) [i_0^n]_\mathsf{tm} : \lambda_n$ is
the iterated identity term on $x_0$. 

\newbox\anglebox
\setbox\anglebox=\hbox{\xy \POS(75,0)\ar@{-} (0,0) \ar@{-} (75,75)\endxy}
\def\angle{\copy\anglebox}

\newbox\coanglebox
\setbox\coanglebox=\hbox{\xy \POS(0,75)\ar@{-} (75,75) \ar@{-} (0,0)\endxy}
\def\coangle{\copy\coanglebox}


\subsubsection{Composition}
\label{sec:composition}
%
For $m>n$ write $s^m_n$, $t^m_n$ for $m-n$ iteration of $s$ and $t$,
respectively. Explicitly:
\begin{align*}
s_n^m &~=~ s_{m-1}\cdots s_n &:~ S_m \Rightarrow S_n\\
t_n^m &~=~ t_{m-1}\cdots t_n &:~ S_m \Rightarrow S_n
\end{align*}
%
For each $n\in \mathbb{N}$ define context $Y_n$ by the pullback in:
\[\bfig
\square/<-`<-`<-`<-/[S_0`S_n`S_n`Y_n;t^n_0`s^n_0`l_n`r_n]
\place(400,100)[\coangle]
\efig\]
%
By definition of pullbacks, $Y_n$ looks like a pair of spans $S_n$
together with the proviso that the variable $1$ of one is always
equal to variable $0$ of the other. I.e. $Y_n$ has the shape of two
spans pasted target-to-source at level $0$. It is easy to check that
this is indeed a pullback. 
%

By the globular identities the two outer squares in the diagram below
commute and by the universal property of the pullback imply a pair
of mediating arrows as indicated. 
\[\bfig
\square/<-`<-`<-`<-/[S_0`S_n`S_n`Y_n;t^n_0`s^n_0`l_n`r_n]
\place(400,100)[\coangle]
\morphism(500,500)/<-/<500,0>[S_n`S_{n+1};s_n]
%\morphism(500,500)/{@{<-}@<.5em>}/<500,0>[S_n`S_{n+1};t_n]
%\morphism(0,0)/{@{<-}@<-.5em>}/<0,-500>[S_n`S_{n+1};s_n]
\morphism(0,0)|l|/<-/<0,-500>[S_n`S_{n+1};s_n]
\morphism(1000,500)|r|/<-/<0,-1000>[S_{n+1}`Y_{n+1};l_{n+1}]
\morphism(0,-500)|b|/<-/<1000,0>[S_{n+1}`Y_{n+1};r_{n+1}]
\place(900,-400)[\coangle]
\morphism(500,0)/<--/<500,-500>[Y_n`Y_{n+1};\langle s , s \rangle_n]
\efig\]
%
Similarly we obtain an arrow $Rr_n : Y_{n+1} \to Y_n$. The morphisms
$l_n$ and $r_n$ provide projections onto the left and right span of
$Y_n$ respectively. The mediating arrows $\langle s, s\rangle_n$ and
$\langle t , t \rangle_n$ provide
projections out of $Y_{n+1}$ onto the join of the sources and targets of the left
and right parts respectively.
%\oxr{We must say a bit more: how is this morphism defined? By recursion. }

In order to define composition we define for each $n$ a third morphism
$c_n: Y_n \Rightarrow S_n$ with the property that both the $s$-squares
and $t$-squares below commute. 
\begin{equation}\label{eq:comp}
\bfig
\square/`<-`<-`/[S_n`S_{n+1}`Y_n`Y_{n+1};`c_n`c_{n+1}`]
\morphism(0,500)|a|/{@{<-}@<.5em>}/[S_n`S_{n+1};s_n]
\morphism(0,500)|b|/{@{<-}@<-.5em>}/[S_n`S_{n+1};t_n]
\morphism(0,0)|a|/{@{<-}@<.5em>}/[Y_n`Y_{n+1};\langle s , s \rangle_n]
\morphism(0,0)|b|/{@{<-}@<-.5em>}/[Y_n`Y_{n+1};\langle t , t \rangle_n]
\efig
\end{equation}
% 
The commutativity of \eqref{eq:comp} expresses the fact that the
source of a composition is a composition of sources and the target of
a composition is a composition of target. 

It follows from all of this that for a context $\Gamma$ and a pair of
morphisms $a , b : \Gamma \Rightarrow S_n$, there is a context morphism $c \langle a , b \rangle
: \Gamma \Rightarrow S_n$ from $s^n_0 a$ to $t^n_0 b$ which is the composition
of $a$ and $b$.
